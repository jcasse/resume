% %%%%%%%%%%%%%%%%%%%%%%%%%%%%%%%%%%%%%%%%%%%%%%%%%%%%%%%%%%%%%%%%%%%%%%%%%%%%%
% set document type and font size
\documentclass[10pt]{article}

% %%%%%%%%%%%%%%%%%%%%%%%%%%%%%%%%%%%%%%%%%%%%%%%%%%%%%%%%%%%%%%%%%%%%%%%%%%%%%
% set font type
% examples of available fonts can be viewed via LibreOffice Writer
\usepackage{fontspec}
\setmainfont{Lato}

% %%%%%%%%%%%%%%%%%%%%%%%%%%%%%%%%%%%%%%%%%%%%%%%%%%%%%%%%%%%%%%%%%%%%%%%%%%%%%
% define one-liner command to declare and set lenght variables
% https://tex.stackexchange.com/questions/
%       172234/define-and-set-length-in-one-command
\newcommand{\deflen}[2]{%
    \expandafter\newlength\csname #1\endcsname
    \expandafter\setlength\csname #1\endcsname{#2}%
}

% %%%%%%%%%%%%%%%%%%%%%%%%%%%%%%%%%%%%%%%%%%%%%%%%%%%%%%%%%%%%%%%%%%%%%%%%%%%%%
% define length variables
\deflen{lmargin}{0.75in}
\deflen{rmargin}{0.75in}
\deflen{tmargin}{0.75in}
\deflen{bmargin}{0.50in}
\deflen{hheight}{\tmargin}
\deflen{hsep}{0.10in}

% %%%%%%%%%%%%%%%%%%%%%%%%%%%%%%%%%%%%%%%%%%%%%%%%%%%%%%%%%%%%%%%%%%%%%%%%%%%%%
% specify paper layout using the geometry package
% https://tex.stackexchange.com/questions/
%       291803/how-to-move-a-header-to-the-extreme-top-of-the-page
%
%     <---------- paper width------------->
%  ^   ------------------------------------
%  |  |    ---------------------  ^        |
%  |  |   |       header        | | top    |
%  |  |    ---------------------  |        |
%  |  |           headsep         V        |
%  p  |    ---------------------     ----  |
%  a  |   |    ^                |   |    | |
%  p  |   |    |   widht =      |   |    | |
%  p  | l | <--- textwidth ---> | r |    | |
%  e  | e |    |                | i |    | |
%  r  | f |    |                | g |    | |
%     | t |    |    body        | h |    | |
%  h  |   |    |                | t |    | |
%  e  |   |    | height =       |   |    | |
%  i  |   |   textheight        |   |    | |<- marginal note
%  g  |   |    |                |   | <->| |-- marginparwidth
%  h  |   |    V                |  <|----|-|-- marginparsep
%  t  |    ---------------------     ----  |
%  |  |           footsep         ^        |
%  |  |    ---------------------  |        |
%  |  |   |       footer        | | bottom |
%  |  |    ---------------------  V        |
%  V   ------------------------------------
%
% includehead and includefoot move the header and footer inside the body
\usepackage[
    letterpaper,
    lmargin=\lmargin,
    rmargin=\rmargin,
    tmargin=\tmargin,
    bmargin=\bmargin,
    headheight=\hheight,
    headsep=\hsep
]{geometry}

% %%%%%%%%%%%%%%%%%%%%%%%%%%%%%%%%%%%%%%%%%%%%%%%%%%%%%%%%%%%%%%%%%%%%%%%%%%%%%
% define custom header and footer
\usepackage{fancyhdr}

\fancypagestyle{firstpageheaderstyle}
{
    % clear header
    \fancyhf{}

    % move header to top edge of paper
    %\setlength{\voffset}{0.00in}

    % strecth header to paper width
    %\addtolength\headwidth{\marginparsep}
    %\addtolength\headwidth{\marginparwidth}
    \fancyheadoffset{\lmargin}

    % draw two horizontal lines (rules) at the top
    % \fancyhead{%
    %     \begin{minipage}{1.0\headwidth}
    %         {\color{headrulecolor}\hrule width \hsize height 3pt}%
    %         {\color{headertoplinecolor}\hrule width \hsize height 3pt}%
    %         \vspace{15pt} % lines vertical position
    %     \end{minipage}
    % }

    % header content
    %\fancyhead[C]{\hspace*{\lmargin}\headerinfo\hspace*{\rmargin}}
    \fancyhead[C]{%
        \raisebox{-0.10\height}{% controls the vertical position of text
            \makebox[0.90\headwidth]{% controls width of text
                \headerinfo%
            }%
        }%
    }

    % set header rule (bottom line)
    % \renewcommand{\headrulewidth}{1pt} % line thickness
    % \renewcommand{\headrule}{%
    %     \vbox to 0pt{\hbox to \headwidth{%
    %         \color{headrulecolor}\leaders\hrule height \headrulewidth\hfill}%
    %     }
    % }

    %\let\oldheadrule\headrule
    %\renewcommand{\headrule}{{\color{headrulecolor}\oldheadrule}}%
    % double lines
    %\renewcommand\headrule{
    %    \begin{minipage}{1\textwidth}
    %        \hrule width \hsize \kern 1mm \hrule width \hsize height 2pt
    %\end{minipage}}%
}
\fancypagestyle{subsequentpageheaderstyle}
{
    % clear header
    \fancyhf{}
    % left corner
    \lhead{Juan Casse}
    % right corner
    \rhead{page \thepage\ of \pageref{LastPage}}
    % hide line (rule)
    \renewcommand\headrulewidth{0pt}
}

% %%%%%%%%%%%%%%%%%%%%%%%%%%%%%%%%%%%%%%%%%%%%%%%%%%%%%%%%%%%%%%%%%%%%%%%%%%%%%
% set bullet size
\renewcommand\labelitemi{\raisebox{0.25ex}{\tiny$\bullet$}}

\usepackage{enumitem}

% interactive rgb color picker:
% https://www.w3schools.com/colors/colors_rgb.asp

% add color to a single table cell: put \cellcolor{blue!25} in cell, example
\usepackage[table]{xcolor}% http://ctan.org/pkg/xcolor

% sahde text with \begin{shaded*}
\usepackage{xcolor}
\usepackage{framed}

% Finally, give us PDF bookmarks
\usepackage{xcolor} % needed to specify urlbordercolor
\usepackage[hidelinks]{hyperref}
\definecolor{darkblue}{rgb}{0.0, 0.353, 0.588}
\hypersetup{%
    colorlinks = false,
    urlbordercolor = darkblue,
    breaklinks,
    linkcolor = darkblue,
    urlcolor = darkblue,
    anchorcolor = darkblue,
    citecolor = darkblue
}
\definecolor{headrulecolor}{rgb}{0.820, 0.820, 0.820} % light grey
\definecolor{headertoplinecolor}{rgb}{0.898, 0.508, 0.234} % flat orange
\definecolor{headerbottomlinecolor}{rgb}{0.820, 0.820, 0.820} % light grey
\definecolor{headerbackgroundcolor}{rgb}{0.950, 0.950, 0.950}

% text justification
\usepackage{ragged2e}

% change line spacing
\usepackage{setspace}

% total number of pages count
\usepackage{lastpage}

%\usefont{OT1}{pcr}{m}{n}

% %%%%%%%%%%%%%%%%%%%%%%%%%%%%%%%%%%%%%%%%%%%%%%%%%%%%%%%%%%%%%%%%%%%%%%%%%%%%%
% Bullets with vertical lines.
% https://tex.stackexchange.com/questions/454744
%     /bulleted-list-with-vertical-lines
% %%%%%%%%%%%%%%%%%%%%%%%%%%%%%%%%%%%%%%%%%%%%%%%%%%%%%%%%%%%%%%%%%%%%%%%%%%%%%
\usepackage{tikz,tikzpagenodes}
\usetikzlibrary{calc}
\usepackage{refcount}

\newcounter{mylist} % new counter for amount of lists
\newcounter{mycnt}[mylist] % create new item counter
\newcounter{mytmp}[mylist] % tmp counter needed for checking before/after current item

\newcommand{\drawoptionsconn}{gray, shorten <= .5mm, shorten >= .5mm, thick}
\newcommand{\drawoptionsshort}{gray, shorten <= .5mm, shorten >= -1mm, thick}

\newcommand{\myitem}{% Modified `\item` to update counter and save nodes
  \stepcounter{mycnt}%
  \item[\linkedlist{%
    i\alph{mylist}\arabic{mycnt}}]%
  \label{item-\alph{mylist}\arabic{mycnt}}%
  \ifnum\value{mycnt}>1%
    \ifnum\getpagerefnumber{item-\alph{mylist}\arabic{mytmp}}<\getpagerefnumber{item-\alph{mylist}\arabic{mycnt}}%
      \begin{tikzpicture}[remember picture,overlay]%
        \expandafter\draw\expandafter[\drawoptionsshort] (i\alph{mylist}\arabic{mycnt}) --
          ++(0,3mm) --
          (i\alph{mylist}\arabic{mycnt} |- current page text area.north);% draw short line
      \end{tikzpicture}%
    \else%
      \begin{tikzpicture}[remember picture,overlay]%
        \expandafter\draw\expandafter[\drawoptionsconn] (i\alph{mylist}\arabic{mytmp}) -- (i\alph{mylist}\arabic{mycnt});% draw the connecting lines
      \end{tikzpicture}%
    \fi%
  \fi%
  \addtocounter{mytmp}{2}%
  \IfRefUndefinedExpandable{item-\alph{mylist}\arabic{mytmp}}{}{% defined
    \ifnum\getpagerefnumber{item-\alph{mylist}\arabic{mytmp}}>\getpagerefnumber{item-\alph{mylist}\arabic{mycnt}}%
      \begin{tikzpicture}[remember picture,overlay]%
      \expandafter\draw\expandafter[\drawoptionsshort] (i\alph{mylist}\arabic{mycnt}) --
        ++(0,-3mm) --
        (i\alph{mylist}\arabic{mycnt} |- current page text area.south);% draw short line
      \end{tikzpicture}%
    \fi%
  }%
  \addtocounter{mytmp}{-1}%
}

\newcommand{\linkedlist}[1]{
  \raisebox{0pt}[0pt][0pt]{\begin{tikzpicture}[remember picture]%
  \node (#1) [gray,circle,fill,inner sep=1.5pt]{};
  \end{tikzpicture}}%
}

\newenvironment{myitemize}{%
% Create new `myitemize` environment to keep track of the counters
  \stepcounter{mylist}% increment list counter
  \begin{itemize}
}{\end{itemize}%
  }
% %%%%%%%%%%%%%%%%%%%%%%%%%%%%%%%%%%%%%%%%%%%%%%%%%%%%%%%%%%%%%%%%%%%%%%%%%%%%%

% %%%%%%%%%%%%%%%%%%%%%%%%%%%%%%%%%%%%%%%%%%%%%%%%%%%%%%%%%%%%%%%%%%%%%%%%%%%%%
% define resume sections

\newcommand{\dividerline}{
    {\color{headerbottomlinecolor}\hrule height 1.0pt \relax}
}

\newcommand{\sectionheader}[1]{
%\vspace{0.5pt}
\begin{flushleft}
\definecolor{shadecolor}{rgb}{0.920, 0.920, 0.920}
\begin{shaded*}
%\vspace{-2.5pt}
%\Large
#1
%\textbf{#1}
%\vspace{-2.5pt}
%\vspace{3.0pt}
%{\color{headerbottomlinecolor}\hrule height 1.0pt \relax}
%{\color{headerbottomlinecolor}\rule{\textwidth}{1.0pt}}
%\dividerline
\end{shaded*}
\end{flushleft}
%\vspace{3.5pt}
}

\newcommand{\afterseparator}[1]{
\begin{justify}
%\vspace{-5pt}
#1
%\vspace{-5pt}
\end{justify}
}

\definecolor{jobcolor}{rgb}{0.0, 0.353, 0.588}
\definecolor{companycolor}{rgb}{0.0, 0.353, 0.588}
\definecolor{educationcolor}{rgb}{0.0, 0.353, 0.588}

\newcommand{\company}[4]{
\begin{justify}
%\vspace{-5pt}  % space above company name
\href{#2}{\textbf{#1}} \\
#3\hfill #4
\dividerline
%\vspace{-5pt}  % space below company info
\end{justify}
}

\newcommand{\dotrule}[1]{\hspace{2pt}\leaders\hbox{$\cdot$}\hfill\hspace{2pt}}

\newcommand{\jobtitle}[2]{
\vspace{-5pt}  % space above job title
\begin{justify}
%\textcolor{jobcolor}{\textsc{#1}}\hfill\textcolor{jobcolor}{\textsc{#2}}
\emph{#1}\hfill\emph{#2}
%\textbf{#1}\dotrule{1pt}\textbf{#2}
\end{justify}
\vspace{-4pt}  % space below job title
}

\newcommand{\education}[5]{
\begin{justify}
\vspace{-5pt}
%\textcolor{educationcolor}{#1}
    \textbf{#1} (GPA #2) \href{#3}{#4} \hfill #5
\vspace{-5pt}
\end{justify}
}

\newcommand{\publication}[5]{
\begin{justify}
\hangindent=15pt
\hangafter=1
#1\ (#2).\ \href{#4}{\textbf{#3}}.\ #5.
\end{justify}
}

% %%%%%%%%%%%%%%%%%%%%%%%%%%%%%%%%%%%%%%%%%%%%%%%%%%%%%%%%%%%%%%%%%%%%%%%%%%%%%
% dot separated list
\deflen{skillswidth}{57pt}
\makeatletter
\newcommand*\commasep[2][\ensuremath{\cdot}]{%
   \begingroup
   \def\commasepsep{\space#1\space}%  store separator
   \def\commasependsep{\space#2\space}%  store last separator
   \@commasep
}
\newcommand*\@commasep[1][]{%
   \def\commasepend{#1}%  store e \hfillnd marker i.e. empty string
   \@ifnextchar\endcommasep{}{%  check for empty list
     \catcode\endlinechar\active%  make end-of-line active
     \commasepfirstelement%  read first element
   }%
}%
\begingroup
\catcode\endlinechar\active%
% The `~` character represents the end-of-line character in the
% code below:
\lccode`\~=\endlinechar%
\lowercase{%
% Read first element
\gdef\commasepfirstelement#1~}{%
    #1%
    \@ifnextchar\endcommasep{%
      \commasepend% remove this if you don't want one with only one element
    }{%
      \commasepelement% no comma here!
    }%
}%
\lowercase{%
\endgroup%
% Read all other elements
\gdef\commasepelement#1~}{%
    \@ifnextchar\endcommasep{% Stop at `\endcommasep`
      \commasependsep#1\commasepend%
    }{%
      \commasepsep#1\commasepelement%
    }%
}%
\def\endcommasep{%
  \@gobble{endcommasep}% be unique (for `\@ifnextchar`)
  \endgroup
}%
\makeatother

% %%%%%%%%%%%%%%%%%%%%%%%%%%%%%%%%%%%%%%%%%%%%%%%%%%%%%%%%%%%%%%%%%%%%%%%%%%%%%
% c++ language symbol
\newcommand{\CPP}{C\texttt{++}\ }
